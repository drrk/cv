\documentclass[10pt]{article}
\usepackage{charter}
\usepackage{fullpage}
\usepackage[resetlabels]{multibib}
\usepackage[ManyBibs,NoDate]{currvita}
% Initialize each paper type for which you need a bibliography.
% Just a dummy parameter is necessary.

% Better for lists with 1-2 items and short descriptions
\newenvironment{sublist}{%
	\begin{list}{}{%
		\setlength{\itemsep}{0em}\setlength{\parsep}{0em}%
		\setlength{\topsep}{0em}\setlength{\parskip}{0em}%
	}%
}%
{ \end{list} }

% Better for lists with more than 2 items and/or long descriptions
\newenvironment{subbulletlist}{%
	\begin{list}{\labelitemii}{%
		\setlength{\topsep}{\itemsep}\setlength{\parskip}{\parsep}%
	}%
}%
{ \end{list} }

\begin{document}

% We'll use this length to change the defaults in some of our lists.
\newlength{\oldcvlabelwidth}
\renewcommand*{\cvbibname}{}

% This is what will appear at the very top of your CV.
\begin{cv}{Kimball Johnson\\{\large \itshape Curriculum Vitae}}

% Items have more vertical space between them than line breaks.
% Note: if you want a linebreak after each cvlist title, you can do:
% \begin{cvlist}{Title \hspace{3in}}
\begin{cvlist}{Contact}
	\item 14 Great Avenham Street\\
	Preston\\
	Lancashire\\
	PR1 3TD
	\item Phone: 01772 592295\\
	Mobile: 07970252372\\
	Email: kimball@bowerham.net
\end{cvlist}
\begin{cvlist}{Education}
	\item \emph{BSc II(ii)(honours) Computer Science}, 2002- 2005\\
	Lancaster University
\end{cvlist}
% cvlist items can take dates as a parameter.
\begin{cvlist}{Employment}

    \item[04/2014-date] Developer
    Shadowcat Systems
    Lancaster, Lancashire
    \item[05/2013-03/2014] Senior Systems Engineer\\
    One Connect Limited\\
    Preston, Lancashire
    \item[02/2011-04/2013] Systems Developer\\
    Lancashire County Council/One Connect Limited\\
    Preston, Lancashire
    \item[07/2007-02/2011] Systems Developer\\
    Lancaster University Network Systems Ltd (LUNS)\\
    Lancaster, Lancashire
    \item[09/2005-07/2007] Software Developer \\
    Auto Execution, Research and Development, Bloomberg LP,
    London
    \item[07/2005-09/2005] Systems Developer \\ 
    Lancaster University Network Systems Ltd (LUNS)\\
    Lancaster, Lancashire
    \item[07/2004-09/2004] Software Developer Internship\\
    Trading Systems, Research and Development, Bloomberg LP,
    London
    \item[05/2004-06/2005] Support Assistant\\
    Lancaster University (Term Time)
    \item[04/2002-09/2002] Audio Visual Technician\\
    University of York
\end{cvlist}
\setlength{\oldcvlabelwidth}{\cvlabelwidth}
\setlength{\cvlabelwidth}{1em}
\begin{cvlist}{Technical Skills}
		\item C++, C, Java, Perl, Ruby, PHP
		\item Chef, Vagrant, Postgresql, MySQL
		\item Agile, SCRUM, JUnit, ITIL, PRINCE2
		\item TCP/IP, Linux, Windows
\end{cvlist}
\setlength{\cvlabelwidth}{\oldcvlabelwidth}
\pagebreak

\setlength{\oldcvlabelwidth}{\cvlabelwidth}
\setlength{\cvlabelwidth}{1em}
\renewcommand*{\bibindent}{1.5em}

\begin{cvlist}{Experience}
    \item \textbf{Shadowcat Systems Limited}
    
    
    On contract to Chronotrack Systems, Evansville, IN, USA

    Chronotrack Systems is a race timing company for marathons and triathlons
    and similar events, they have a number of parts, from providing the
    hardware to chip-time events, and online event booking and scoring system,
    and a social network for athletes.

    I was brought on board into the Systems Administration Team to provide
    cover for European hours, and to be part of the effort to improve and
    modernise the infrastructure.  All the infrastructure ran on Amazon
    Webservices but there was very little in the way of configuration
    management.  I assisted in bringing more structure to the systems, and
    took the lead on developing a new Continuous Integration System for the
    developers, to improve the existing time to starting tests from over an
    hour, to approximately 5 minutes.  

    I was also key in the redesign of the developer database system, moving
    from a system where by the production database was dumped daily and then
    restored from fresh onto the developer database, where the developers
    could access the daily snapshot, to a much faster, and less disk intensive
    system where the database is continously replicated from production, and point
    in time snap shots can be taken on demand, and are automatically taken every
    few hours.

    At Chronotrack we employed a number of agile methodoliges, including
    KanBan and SCRUM. 
    \item \textbf{Lancashire County Council/One Connect Limited}

    One Connect Limited (OCL) is a Strategic Partnership between BT and Lancashire
    County Council. In 2011 the contract from LUNS was in-sourced, and I was TUPEd 
    and seconded into OCL.

    At OCL I am part of the Data Centre Infrastructure team, responsible for
    all the Unix and Linux systems, the virtual machine infrastructure and the
    network storage infrastructure.  Since moving from LUNS I have driven a
    project to bring configuration management into the systems, as the systems
    at LUNS were heavily dependent upon it.  The system in use is Opscode
    Chef. One of my main roles has been leading the roll-out of Chef across the
entirety  of the systems at the council.  We have started with the systems in
the education section, initially focusing on ensuring that users are
consistent on all systems, and that we can quickly remove accounts for staff
that leave. In addition, I have been tasked with writing and implementing policy for
all new systems that are built, focusing on Linux based systems, in order to
make it easier for new and existing staff to understand the setups.   I have
also been assisting in the training of the other members of the team in using
Chef, and in writing cookbooks for it.

    As a result of the transfer from LUNS we also took on a number of the
systems that were in use there.  I have retained primary responsibility for the
systems listed below, with the exception of the Cumbrian Shared Hosting which
 remained in the control of LUNS.  As we also took on the Moodle VLE
environment, which is used by approximately 250,000 pupils, I have also taken
over the infrastructure management of that, and have been the primary technical
designer in the project to refresh the platform, and bring it up to the latest
version.  This has involved working with other teams in web-services,
networking and project management, as well as the senior management team and
external suppliers.  
     
    In my more recent role as Senior Systems Engineer, I report directly to
the Platform Architect for Storage and Unix based systems, and I am responsible
for all the new infrastructure that we are tasked to build by the project
design team.  This in practice includes early involvement with the design team
to ensure the best solution is produced.
	
	\item \textbf{LUNS}

    LUNS's primary business was a contract with the Cumbria and Lancashire
    Education Online (CLEO) Regional Broadband Consortium.  This contract was
    to provide all of the network systems, and a large proportion of the
    central computer systems for the schools in the two counties.
    Additionally they had a separate contract with Cumbria County Council to
    provide further central software services for Cumbrian schools. All projects 
	were managed using PRINCE2 and service was managed under the ITIL v3 Best Practices.
	\begin{subbulletlist}
    \item Network Edge Device

        One of my main projects at LUNS was the ongoing development of the
        Network Edge Device that is installed into every school in the two
        counties.  This device, based on Debian GNU/Linux, is for local 
        web caching, filtering and content provision.  It is designed to 
        be centrally managed, such that it can recover from almost any
        corruption or configuration problems simply with a reboot. I have 
        been involved with all aspects of this project, particularly with the 
        database system, and stored procedures. Another major aspect of this 
        project has been integration of the web proxy authentication with
        Microsoft Active Directory, initially through NTLM, and most recently
        Kerberos. 
    \item Dashboard Management system

		As part of the NED project described above, there was need for a central 
		web based management system.  To this end we designed a modular PHP5 class 
		based framework, and a hierarchical data structure. I was greatly involved 
		in this system, and wholly responsible for the database development. This was 
		developed in PostgreSQL with PL/Perl stored procedures.
    \item Video Conferencing 

        CLEO provide a video conferencing service to schools and museums in
        both counties, based on the GnuGK gatekeeper software.  I have
        been the lead on this service since I started at LUNS full time, and
        have been responsible for a migration of the management and
        authentication system from the old, hard to manage system, to integrate
        with the dashboard configuration system, used by the NED project.
    \item Cumbrian Shared Hosting 

        Another major project was the re-development of the Cumbrian
        schools shared hosting solution.  I was the project lead for this,
        organising the work for myself and my colleagues to produce a new 
        secure hosting solution for the schools.  I designed a 
        solution that takes advantage of separation of concerns with Linux 
        Virtual Servers, and by employing various security features to help
        prevent spread of any attacks or exploits from one site to another.
	\end{subbulletlist}
	
	\item \textbf{Bloomberg}

	\begin{subbulletlist}
    \item E-Bond Trading Platform

    After the 10 week training class for the full time position, I became part of 
    the Auto Execution applications development team, initially working on the 
    E-Bond electronic trading platform. I quickly took over as the main support 
    contact for the back end server programs running the application, and 
    additionally developed many new features as the project phases continued.
    This was developed in C and C++, and GTK for the User Interface.  This low latency 
	real-time system was in use at the time as the canonical trading platform for 
	many emerging bond markets.
    
    \item T-Zero Integration

    The next major project I worked on was integration with the T-Zero
    Credit Default Swap trade reporting system. For this greenfield project 
	I designed the back-end multi-threaded server application from scratch to integrate 
	with their Java Enterprise application server.  As this systems was developed
	in Java it provided a great challenge integrating new technologies with the dated 
    existing Bloomberg Systems.  
	\end{subbulletlist}


\end{cvlist}

\pagebreak
\begin{cvlist}{Extracurricular}
\item \textbf{FLOSS UK}

Since 2008 I have been a member of FLOSS UK (then known as UKUUG) and attended
a number of conferences organised by them.  In September 2010 I was elected
onto the council, and then took over the running of the spring conference in
2012.  I was then elected as Chairman of FLOSS UK in September 2012 and I 
continue to take the primary role in running the conference events.

\item \textbf{Hackspaces}

I am a on the board of HacMan Manchester, acting as their treasurer. I have a 
keen interest in electronics, especially embedded
development.  Currently one of the other members of HacMan and I are building
an automated model railway shunting puzzle solver. In the spirit of hack
spaces, I also make efforts to help other members that have less experience.

\end{cvlist}
\end{cv}
\end{document}

